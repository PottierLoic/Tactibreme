\documentclass{article}
\usepackage{graphicx}

\title{Doc technique}

\begin{document}

\maketitle

\section{Introduction}


\section{Combien d'IA}
une seule pour tout ?
vraiment different truc et formats d'entree/sortie pour init le jeu et jouer ducoup yen a 2

\subsection{Jeu}
Début : une IA qui joue contre elle meme -> stagne (donner des data)
Le probleme -> elle change de place pour jouer contre elle meme ducoup elle a des reward pour ladversaire.
-> il en faut deux

\subsection{Placement des pieces}
Faire une ia en plus

\section{Comment savoir combien de couches}
optuna, hyperopt...

jeu:
2 ia rouge vs bleu qui joue toujours du meme coté
2 ia rouge vs bleu qui alterne le coté - on normalise le plateau (retourne le plateau + inverse couleurs)
1 ia qui joue contre lui meme -> stagne

recompences:
moves invalides acceptés -> malus
filtrage des moves invalides (recompense que victoire)
recompenser des moves precis (rempiler 3 ...)

opti:
paralleliser l'entrainement
batcher les inputs/outputs
l'idee de loïc

\section{man avec les record + train}

\section{fichiers crees}
csv des moves
logs
checkpoint
fichier de stats

\section{Trucs réalisés}
jeu
Ia qui joue
truc qui analyse les games déja jouees
ui
ia de placement

\end{document}
